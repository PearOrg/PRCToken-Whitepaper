\chapter{Pair with Pear, Share Everywhere}
区块链技术在世界飞速发展,堪称互联网界的技术革命,各国都已加入区块链“军备竞赛”,各行各业也都纷纷迎合技术发展趋势,寻求与区块链的结合点,区块链技术与产业呈良好的发展态势。\par
区块链技术的广泛应用,必将加速数字化信用社会的到来,但是正如前文所述,区块链的去中心化并不是无中心化,未来区块链必将出现大量的“多极”体系。区块链技术满足了人权的的自由与信任的诉求,是其炙手可热的最根本原因,互联网的开放包容共享的环境,是区块链茁壮成长的保证。\par 
现阶段区块链是处在探路的前期,缺少统一的标准与有效的监管,是最为艰难,也是最鱼龙混杂的时期。似乎无论什么与区块链相沾边都可以引起万人空巷的抢购,共享用户带宽、计算、存储资源的公司也继续延续着大多数共享企业的商业战略,抢占用户,每家公司其他同行划清界限,用户往往疲于抢购完这家设备再去抢购再去抢购另外一家。到头下来,设备买了一堆,但是一个用户可以共享的资源就那么些,一条网线连接这四五家价值几千到上万不等的高昂设备,各设备的收益计算各不相同,互相抢占影响用户总收益。\par 
这一批互联网公司违背了共享精神,违背了区块链理念。共享经济是为了将闲置的资源,产能过剩的资源拿出来共享,而绝非每家制造独立的硬件诱导用户过度购买。区块链技术是为了去中心化,为的是分散的集权,所有用户一起建立的信用背书,而绝非一家独大,占据全部的用户节点,掌握行业的话语权,也绝非每家都相互独立,让用户无所适从。\par
为了贯彻落实习总书记“一花独放不是春,百花齐放满园春”的指示,践行Pear的让分享无处不在的理念,Pear将做到:\par
1.融合创新,开放包容。\par
Pear将一直探索创新区块链与共享带宽、存储、计算资源与区块链技术的结合,保证自身始终站立在整个行业的最前沿,以身作则,引领整个行业的健康稳定发展。同时积极与处于生态链中的各个厂商积极合作,以开放包容的态度进行各行业间的交流与和合作,共同探索发展道路。\par 
2.不忘初心,不断提升自身技术核心优势。\par 
Pear作为共享雾计算的开创者与践行者,将时刻铭记自身使命,发扬共享精神,坚决杜绝抵制硬件产能过剩导致资源浪费的情形,将用户手中已有资源和设备进行打通,同时与支持和接受我们理念的硬件厂商进行合作,从源头开始散播共享理念。将自身的服务场景与应用进行全网全设备的打通,做到,每台网络设备都可以进行共享,每台进行共享的设备都有真实的服务场景,每个真实的服务场景都能产生社会效益。\par
3.共享共赢,健康发展。\par 
Pear接受任何可能的合作方式与合作伙伴。同时推行所有合作伙伴中用户的互通,用户手中的贡献证明只要是依据Pear提供给合作伙伴相关服务产生的,便可以实现全平台的流通,实现真正的跨设备,跨平台,跨网络的共享共赢,用户拥有任意一家合作伙伴的路由、NAS、家庭网关、机顶盒、电脑、物联网等网络设备,即可接入Pear Fog 平台。只需拥有一台设备即可享受全平台的服务,这是除Pear外任何一家打着共享带宽、存储、计算资源旗号的公司不可能,也不愿意做到的。Pear愿意与大家一起建设发展互联网共享经济,区块链技术。\par 
和梨享一起,让分享无处不在!Pair with Pear,Share Everywhere!