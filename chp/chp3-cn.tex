\chapter{合作案例——民生链}
在CDN内容共享平台搭建日趋成熟的基础之上,民生链项目团队将围绕着“构建信任、
智能、⾼效的新共享经济”的目标,继续提升区块链技术的上升空间,丰富智能应用,打造
一个TSD(TRUST-SECURITY-DEVELOPMENT)智能生态。
\section{民生币}
民生币是Pear依据合作方需求,为其定制的一整套解决方案
\subsection{民生币介绍}
民生币是民生链项目产生的虚拟数字证明。用户共享带宽、流量等闲置资源之后,团队
会根据共享量的权重进行民生币的分发。民生币是基于区块链技术设计产生的,并且仅能通
过区块链路由矿机设备共享资源后获取。\par 
每一位使用区块链路由矿机设备共享带宽资源的用户,所获得的民生币均存储于安全的
区块链中,每一位用户都拥有独立账号和隐私密码,可以对这些数字证明进行有效操作。
\subsection{民生币应用场景}
用户获得民生币后,可以使用民生币在TSD 智能共享生态里获得增值服务,包括:\par 
观影加速:电影、电视剧、电视直播等视频在线播放速度提升;\par 
网络传输加速:电影、电视等资源下载、软件更新、软件下载、文件传输等速度提升;\par 
网页加速:电商频道、个站、论坛社区等网页加载速度提升等。
\subsection{民生币的发行与分配}
民生币采用POS(Proof of stake)算法,取决于设备硬件能力、上传流量、存储大小、在线
时长等。
每一个发币期$t$内,总有效在线节点数为$n$,总发币量为$c$,按硬件能力、存储能力、流量贡献分以下若干部分发行:
  $$c = c_h + c_s + c_b + c_d$$

记四者所占权重:$w_h = \frac{c_h}{c}, w_s = \frac{c_s}{c}, w_b = \frac{c_b}{c}, w_d = \frac{c_d}{c}$,显然$w_h + w_s + w_b + w_d = 1$。初期取值可为:$w_h = 0.2, w_s = 0.3, w_b = 0.2,  w_d = 0.3$。

节点$i$获得的币量$c(i)$亦由相应四部分构成:
$$c(i) = c_h(i) + c_s(i) + c_b(i) + c_d(i)$$\\
 $c_h(i)$为节点$i$的硬件能力值$p_h(i)$占所有节点硬件能力值之和$\sum_{k=1}^n{p_h(k)}$的比例决定;\\
$c_s(i)$为节点$i$的存储能力值$p_s(i)$占所有节点存储能力值之和$\sum_{k=1}^n{p_s(k)}$的比例决定;\\
$c_b(i)$为节点$i$的带宽能力值$p_b(i)$占所有节点带宽能力值之和$\sum_{k=1}^n{p_b(k)}$的比例决定;\\
$c_d(i)$为节点$i$的流量贡献值$p_d(i)$占所有节点流量贡献值之和$\sum_{k=1}^n{p_d(k)}$的比例决定:\\
$$c_{\*}(i) = c_{\*}\times \frac{p_{\*}(i)}{\sum\limits_{k=1}^n{p_{\*}(k)}}$$

对于$p_h$,可挖数字证明的配置最低的硬件为基准值1。\\

对于$p_s$,
 $$p_s = \log{(\frac{s}{250}+1)}$$
其中$s$为节点可用于挖矿的存储空间的一个测度:$s = s_u + \frac{1}{2}s_a$,其中$s_u$为已用于挖矿的空间,$s_a$为可用于挖矿但未使用的空间,单位均为GB。\\

对于$p_b$,取上行带宽$b_u$和下行带宽$b_d$的较小值,即$\min(b_u, b_d)$。\\

用户贡献流量每5min统计一次,单位字节,每天共288个值,记为一个长度为288的向量$\vec{d}$。在节点所处的地域/时区,每个5min所对应时间段的流量价值是不一样的,记为向量$\vec{v}$。那么:
$p_d = \vec{d} \cdot \vec{v}$
