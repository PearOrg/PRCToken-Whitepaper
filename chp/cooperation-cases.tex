\chapter{Cooporation Cases}\label{chp-cooporation-cases}
Based on the increasingly mature CDN content sharing platform, the Gehua Chain project team will focus on ``building a trustable, smart, and effective new shared economy'', continuously to exploit the blockchain technology capabilities and enrich the intelligent applications, and build a TSD  (Trust-Security-Development) smart ecology.

\section{Gehua Chain}
Gehua Chain is one of Pear's tailored set of solutions according to the needs of its partners. 

The Gehua Token is a virtual digital proof generated by the Gehua Chain project. Via sharing bandwidth, storage, compute, and other idle resources, Gehua Chain router users can mine tokens issued by the Gehua team. The number of tokens a user receives is according to his/her weight of contribution. 
All the proofs are stored in secure Blockchain. Each of the Gehua Chain router miner users has an account, a password or certificate, and can take effective operations on top of his/her proofs.

\subsection{Use Scenarios of the Gehua Token}
The Gehua Token works as the exchange medium for the TSD smart sharing ecosystem in two ways: 
\begin{enumerate}
	\item It is used by many content providers to purchase TSD services ({\em e.g.} Fog CDN services). As such, we define it as a utility token. 
	\item Miners can use it to exchange for digital contents or other digital services ({\em e.g.} membership fees). Here the Gehua Token is a currency token as it is a medium of value exchange. 
\end{enumerate}
The Gehua token is not an asset token, therefore it does not offer any dividends profits or voting right. While there will be direct or indirect ways to exchange Gehua Token to or from other digital cryptocurrencies or fiat currencies, the main purpose of Gehua Token is to be the utility token within the Gehua Chain ecosystem. 

\subsection{Distribution of the Gehua Token}
The Gehua Token adopts the PoS (Proof of Stake) algorithm, taking the device's hardware (compute and memory) capability, bandwidth capability, storage capability, traffic contribution, and online duration into account. 

In each digital PoS period $t$, $n$ is the total number of valid online nodes, $c$ is the total amount of tokens to be issued, then $c$ is divided into 4 parts: hardware capability, bandwidth capability, storage capability, traffic contribution: 
\begin{equation}
	c = c_h + c_s + c_b + c_d
\end{equation}

We denote their weights as: $w_h = \frac{c_h}{c}, w_s = \frac{c_s}{c}, w_b = \frac{c_b}{c}, w_d = \frac{c_d}{c}$, apparently $w_h + w_s + w_b + w_d = 1$. Here is a sample tuple of values: $w_h = 0.2, w_s = 0.3, w_b = 0.2,  w_d = 0.3$. 

The digital certificate $c(i)$ obtained by node $i$ consists of 4 parts:
\begin{equation}
	c(i) = c_h(i) + c_s(i) + c_b(i) + c_d(i)
\end{equation} 

$c_{*}(i)$ is decided by the ratio of the current total circulation in one category $c_{*}$ multiplied by the power value $p_{*}(i)$ of the node $i$ to the sum of the corresponding capability/contribution values of all nodes: 
\begin{equation}
	c_{*}(i) = c_{*}\times \frac{p_{*}(i)}{\sum\limits_{k=1}^n{p_{*}(k)}}
\end{equation}

For $p_h$, we set the configuration of the first generation router, as the minimum configuration of hardware to mine the Gehua Token, and we assign it the benchmark value of 1. 

For $p_s$, 
\begin{equation}
	p_s = \log{(\frac{s}{S}+1)}
\end{equation}
where $s$ is a measure for the storage space available for mining, and 
\begin{equation}
	s = s_u + \lambda s_a
\end{equation}
where $s_u$ is the space that has already been used for mining, and $s_a$ is the unused space available for mining, $S$ is a measure adjustment parameter and its initial desirable is 250, all in GB. Following the principle of maximum entropy, we set $\lambda = \frac{1}{2}$. 

For $p_b$, we take the minimal of uplink bandwidth $b_u$ and the downlink bandwidth $b_d$, namely $\min(b_u, b_d)$. 

We calculate the Internet data traffic that each node contributes every five minutes and we get a total of 288 values a day. We record it as a vector $\vec{d}$ of length 288. In the region/time zone where the node is located, the value of Internet data traffic of each time slot corresponding to 5 minutes is not the same, and is denoted as vector $\vec{v}$. Then we have: 
\begin{equation}
	p_d = \vec{d} \cdot \vec{v}
\end{equation}

We can set maximum values for $p_h, p_s, p_b, p_d$, no additional mining reward a node can earn if any of its power value exceeds a given one. The weights $w_h, w_s, w_b, w_d$, $S$ and $\lambda$ will also be voted on, updated and publicised through a committee constitutes of three parties: miners, fog service clients, and the Gehua team. 