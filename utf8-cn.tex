% LaTeX-chinese-template v0.4
% https://github.com/CTeX-org/ctex-kit/
%
% You may need to check the instructions at the end of this file.
% See the manual for the collection CTeX for more details.
%
\documentclass[UTF8, zihao=-4, a4paper, fancyhdr, fntef]{ctexrep}
\usepackage{fancyhdr}
\usepackage{url,hyperref}
\usepackage[top=3cm, bottom=2.5cm, left=2.5cm, right=2cm, head=0.5cm, headsep=0.5cm, foot=0.75cm]{geometry}
\CTEXsetup[number={\arabic{chapter}}, beforeskip={0pt}]{chapter} %format={\zihao{-3}\centering
\CTEXoptions[contentsname={目\quad 录}]

% Alan: begin the font trial
% Euler for math | Palatino for rm | Helvetica for ss | Courier for tt
\renewcommand{\rmdefault}{ppl} % rm
%\linespread{1.05}        % Palatino needs more leading
\usepackage[scaled]{helvet} % ss
\usepackage{courier} % tt
%\usepackage{euler} % math
\usepackage{eulervm} % a better implementation of the euler package (not in gwTeX)
\normalfont
\usepackage[T1]{fontenc}
% Alan: end the font trial

\newcommand\specialchapter[1]{
    \chapter*{#1}\addcontentsline{toc}{chapter}{#1}
    \markboth{#1}{}\phantomsection}

\renewcommand{\headrulewidth}{0.5pt}
\pagestyle{fancy}
\fancyhead[l]{{\includegraphics[scale=.025]{PearLogo.eps}~}\fancyplain{}{\sl\rightmark}}
%\lhead{\zihao{5}\includegraphics[scale=.025]{PearLogo.eps}~白皮书}
%\fancyhead[r]{\zihao{5}\fancyplain{}{\sl\leftmark}}
%%\fancyhead[H]{\zihao{5}}
%\fancyfoot[c]{\zihao{-5}\rm\thepage}

\newcommand*{\keywords}[1]{\noindent\textbf{Keywords: }#1}
\newcommand*{\cnkeywords}[1]{\noindent\textbf{关键词: }#1}

\usepackage{mathpazo,amsmath,amsfonts,amssymb,epsfig,enumerate,bbm,calc,color,ifthen,capt-of} % original was times, but I think it's ugly; we use the same as IEEE CompSoc
\usepackage{algorithm,algpseudocode}
\usepackage{hyperref}
\PassOptionsToPackage{hyphens}{url}
%\usepackage{breakurl}
\usepackage{geometry}
\usepackage{array}
\usepackage{forest}
\usepackage[absolute,overlay]{textpos}
\usepackage[center]{subfigure}
\usepackage{color,graphicx}
\usepackage{multirow}
\usepackage{booktabs,threeparttable}
\usepackage[formats]{listings}

\usepackage[font=small,justification=centering]{caption}
%\usepackage[font=footnotesize]{subcaption}

\def\UrlBigBreaks{\do\/\do\-\do:\do\_\do\#\do\?}

\usepackage{makecell}
\usepackage{tabularx}


\usepackage{pgfplots,pgfplotstable}
\usepackage{sfmath}
\usepackage{xcolor}

\usepackage{listings}

\usepackage{tablefootnote}

\renewcommand\bibname{References}
\renewcommand{\vec}[1]{\mathbf{#1}}

\usepackage{tikz}
\usetikzlibrary{arrows,shapes,positioning,shadows,trees,fit,calc}
\tikzset{font=\footnotesize}
\tikzset{
  basic/.style  = {draw, rectangle},
  root/.style   = {basic, rounded corners=2pt, thin, align=center,
                   fill=green!30},
  level 2/.style = {basic, rounded corners=6pt, thin,align=center, fill=green!60},
  level 3/.style = {basic, thin, align=left, fill=pink!60}
}

\tikzset{font=\footnotesize}

\definecolor{mygreen}{rgb}{0,0.6,0}
\definecolor{mygray}{rgb}{0.5,0.5,0.5}
\definecolor{mymauve}{rgb}{0.58,0,0.82}
\lstset{ %
	float=ht,
	backgroundcolor=\color{white},   % choose the background color; you must add \usepackage{color} or \usepackage{xcolor}
	basicstyle=\footnotesize\ttfamily,        % the size of the fonts that are used for the code
	breakatwhitespace=false,         % sets if automatic breaks should only happen at whitespace
	breaklines=true,                 % sets automatic line breaking
	captionpos=b,                    % sets the caption-position to bottom
	commentstyle=\color{mygreen},    % comment style
	deletekeywords={...},            % if you want to delete keywords from the given language
	escapeinside={(*@}{@*)},          % if you want to add LaTeX within your code
	extendedchars=true,              % lets you use non-ASCII characters; for 8-bits encodings only, does not work with UTF-8
	frame=tb,	                   % adds a frame around the code
	keepspaces=true,                 % keeps spaces in text, useful for keeping indentation of code (possibly needs columns=flexible)
	keywordstyle=\color{blue},       % keyword style
	%language=C,                     % the language of the code
	otherkeywords={...},           % if you want to add more keywords to the set
	numbers=left,                    % where to put the line-numbers; possible values are (none, left, right)
	numbersep=5pt,                   % how far the line-numbers are from the code
	numberstyle=\tiny\color{mygray}, % the style that is used for the line-numbers
	rulecolor=\color{black},         % if not set, the frame-color may be changed on line-breaks within not-black text (e.g. comments (green here))
	showspaces=false,                % show spaces everywhere adding particular underscores; it overrides 'showstringspaces'
	showstringspaces=false,          % underline spaces within strings only
	showtabs=false,                  % show tabs within strings adding particular underscores
	stepnumber=2,                    % the step between two line-numbers. If it's 1, each line will be numbered
	stringstyle=\color{mymauve},     % string literal style
	tabsize=2,	                   % sets default tabsize to 2 spaces
	%title=\centering\lstname                   % show the filename of files included with \lstinputlisting; also try caption instead of title
}



\title{\includegraphics[scale=.25]{PearLogo.eps} \\Pear区块链白皮书}
\author{江英豪\\
		\href{mailto:j@pear.hk}{\nolinkurl{(j@pear.hk)}}
		}

\begin{document}

\maketitle
\pagenumbering{roman}

\specialchapter{前\quad 言}
区块链技术给互联网数字经济时代带来了巨变的曙光。\par
这种巨变在互联网近50年的历史上曾发生过两次。第一次巨变是全球性的联网,自1969年阿帕网诞生以来,全世界主流国家逐渐接入互联网,开启了全球联网的征程。第二次巨变是全球性的应用,自1989年万维网论文问世后,互联网应用全面开花,实现了应用全球爆发。\par
第三次巨变正在酝酿。2009年比特币诞生是个标志性事件。在区块链技术的支持下,比特币打破了传统纸币的“暗黑”盒子。作为实体的纸币的流通是看不见的,没有人知道一张纸币从哪里来到哪里去,而区块链却可以让数字货币的每一笔动向都清清楚楚有“链”可查,同时还可以保护参与者的隐私。\par
人们发现,区块链的意义在于可以构建一个更加可靠的互联网系统,从根本上解决价值交换与转移中存在的欺诈和寻租现象。越来越多的人相信,随着区块链技术的普及,数字经济将会更加真实可信,经济社会由此变得更加公正和透明。\par
进一步的研究发现,区块链技术具备一种“降低成本”的强大能力,能简化流程,降低一些不必要的交易成本及制度性成本。这种能力应用于许多社会领域中,对于改善当前低迷的经济环境更有现实意义。\par
区块链引发了世界性的关注,迅速地成为一场全球参与竞逐的“军备”大赛,许多国家认识到区块链技术巨大的应用前景,开始从国家层面设计区块链的发展道路。\par
2017年,区块链及相关行业加速发展,全球正在跑步进入“区块链经济时代”。在全球范围内,会出现更多的成熟应用。此时此刻,中国面临重大机遇。\\
\\
\rightline{摘自《腾讯区块链方案白皮书》}

\vspace{12pt}
%\cnkeywords{内容分发网络,网络文件系统,回源,淘汰,删除,目录遍历,并行,幂律分布}

%\specialchapter{Abstract}


%This document was typesetted and compiled with the ~\LaTeX~ template by my own. Anyone senior who is interested

%in writing technical/academic reports with it can acquire a copy from me at any time (though there may exist some boring bugs).


%\vspace{12pt}

%\keywords{CDN, NFS, Back to Source, Elimination, Deletion, Directory Traversal, Parallelism, Power-law Distribution}

\tableofcontents
\pagenumbering{arabic}

\input{chp/chp1}
\input{chp/chp2}
\chapter{合作案例——民生链}
在CDN内容共享平台搭建日趋成熟的基础之上,民生链项目团队将围绕着“构建信任、
智能、⾼效的新共享经济”的目标,继续提升区块链技术的上升空间,丰富智能应用,打造
一个TSD(TRUST-SECURITY-DEVELOPMENT)智能生态。
\section{民生币}
民生币是Pear依据合作方需求,为其定制的一整套解决方案
\subsection{民生币介绍}
民生币是民生链项目产生的虚拟数字证明。用户共享带宽、流量等闲置资源之后,团队
会根据共享量的权重进行民生币的分发。民生币是基于区块链技术设计产生的,并且仅能通
过区块链路由矿机设备共享资源后获取。\par 
每一位使用区块链路由矿机设备共享带宽资源的用户,所获得的民生币均存储于安全的
区块链中,每一位用户都拥有独立账号和隐私密码,可以对这些数字证明进行有效操作。
\subsection{民生币应用场景}
用户获得民生币后,可以使用民生币在TSD 智能共享生态里获得增值服务,包括:\par 
观影加速:电影、电视剧、电视直播等视频在线播放速度提升;\par 
网络传输加速:电影、电视等资源下载、软件更新、软件下载、文件传输等速度提升;\par 
网页加速:电商频道、个站、论坛社区等网页加载速度提升等。
\subsection{民生币的发行与分配}
民生币采用POS(Proof of stake)算法,取决于设备硬件能力、上传流量、存储大小、在线
时长等。
每一个发币期$t$内,总有效在线节点数为$n$,总发币量为$c$,按硬件能力、存储能力、流量贡献分以下若干部分发行:
  $$c = c_h + c_s + c_b + c_d$$

记四者所占权重:$w_h = \frac{c_h}{c}, w_s = \frac{c_s}{c}, w_b = \frac{c_b}{c}, w_d = \frac{c_d}{c}$,显然$w_h + w_s + w_b + w_d = 1$。初期取值可为:$w_h = 0.2, w_s = 0.3, w_b = 0.2,  w_d = 0.3$。

节点$i$获得的币量$c(i)$亦由相应四部分构成:
$$c(i) = c_h(i) + c_s(i) + c_b(i) + c_d(i)$$\\
 $c_h(i)$为节点$i$的硬件能力值$p_h(i)$占所有节点硬件能力值之和$\sum_{k=1}^n{p_h(k)}$的比例决定;\\
$c_s(i)$为节点$i$的存储能力值$p_s(i)$占所有节点存储能力值之和$\sum_{k=1}^n{p_s(k)}$的比例决定;\\
$c_b(i)$为节点$i$的带宽能力值$p_b(i)$占所有节点带宽能力值之和$\sum_{k=1}^n{p_b(k)}$的比例决定;\\
$c_d(i)$为节点$i$的流量贡献值$p_d(i)$占所有节点流量贡献值之和$\sum_{k=1}^n{p_d(k)}$的比例决定:\\
$$c_{\*}(i) = c_{\*}\times \frac{p_{\*}(i)}{\sum\limits_{k=1}^n{p_{\*}(k)}}$$

对于$p_h$,可挖数字证明的配置最低的硬件为基准值1。\\

对于$p_s$,
 $$p_s = \log{(\frac{s}{250}+1)}$$
其中$s$为节点可用于挖矿的存储空间的一个测度:$s = s_u + \frac{1}{2}s_a$,其中$s_u$为已用于挖矿的空间,$s_a$为可用于挖矿但未使用的空间,单位均为GB。\\

对于$p_b$,取上行带宽$b_u$和下行带宽$b_d$的较小值,即$\min(b_u, b_d)$。\\

用户贡献流量每5min统计一次,单位字节,每天共288个值,记为一个长度为288的向量$\vec{d}$。在节点所处的地域/时区,每个5min所对应时间段的流量价值是不一样的,记为向量$\vec{v}$。那么:
$p_d = \vec{d} \cdot \vec{v}$

\input{chp/chp4}
%\bibliographystyle{IEEEtran}  % 2nd choice: plain
%\bibliography{IEEEabrv,ref}
%\addcontentsline{toc}{chapter}{参考文献}
\end{document}

----- How to Compile -----

This LaTeX manuscript should be saved as UTF-8 encoding, and be compiled by

* LaTeX - DVIPDFMx or,
* pdfLaTeX or,
* XeLaTeX or,
* LuaLaTeX.

Note that, the later two compiling method are recommended.

----- Fonts -----

The collection CTeX at version 2.0 or later has the ability to detect what
kind of operation system that you are using. As a result, the collection will
automatically choose an appropriate fontset for the specific OS.

* Mac OS X - SinoType(华文) will be selected
* Windows  - Sim(中易) will be selected
* Others   - Fandol, which is distributed alone with TeX Live, will be selected

Since the fontset SinoType couldn't work under pdfTeX, you may only use XeLaTeX
and/or LuaLaTeX on Mac OS X. Otherwise, you will have to choose a fontset
manually.

----- Environment -----

Tested with TeX Live 2015 on Mac OS X (10.9.5).

* xetex       r37058      3.14159265-2.6-0.99992 (TeX Live 2015)
* pdftex      r37159      3.14159265-2.6-1.40.16 (TeX Live 2015)
* luatex      r37078      Version beta-0.80.0 (TeX Live 2015) (rev 5228)

* CTeX        5d27e06     v2.0 2015/05/08
* CJK         r36951      v4.8.4
* xeCJK       r35946      v3.3.0
* LuaTeX-ja   r36992      20150420.0

Hope these information useful for TeX newbies :-)
  Liu Yubao <yubao.liu at gmail dot com>

ChangeLog:
  2009-09-11  Liu Yubao
      * initial version, v0.1

  2009-09-13  Liu Yubao
      * use zhwinfonts to embed truetype fonts with pdflatex and dvipdfmx
      * set hyperref properly for different encodings and TeX engines to avoid garbled bookmarks
      % release v0.2

  2010-06-20  Liu Haiyang
      * simpify zhwinfonts and hyperref, v0.3

  2015-05-08  Liam Huang
      * CTeX 2.0 is about to release
      * v0.4