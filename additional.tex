\usepackage{mathpazo,amsmath,amsfonts,amssymb,epsfig,enumerate,bbm,calc,color,ifthen,capt-of} % original was times, but I think it's ugly; we use the same as IEEE CompSoc
\usepackage{algorithm,algpseudocode}
\usepackage{hyperref}
\PassOptionsToPackage{hyphens}{url}
%\usepackage{breakurl}
\usepackage{geometry}
\usepackage{array}
\usepackage{forest}
\usepackage[absolute,overlay]{textpos}
\usepackage[center]{subfigure}
\usepackage{color,graphicx}
\usepackage{multirow}
\usepackage{booktabs,threeparttable}
\usepackage[formats]{listings}

\usepackage[font=small,justification=centering]{caption}
%\usepackage[font=footnotesize]{subcaption}

\def\UrlBigBreaks{\do\/\do\-\do:\do\_\do\#\do\?}

\usepackage{makecell}
\usepackage{tabularx}


\usepackage{pgfplots,pgfplotstable}
\usepackage{sfmath}
\usepackage{xcolor}

\usepackage{listings}

\usepackage{tablefootnote}

\renewcommand\bibname{References}
\renewcommand{\vec}[1]{\mathbf{#1}}

\usepackage{tikz}
\usetikzlibrary{arrows,shapes,positioning,shadows,trees,fit,calc}
\tikzset{font=\footnotesize}
\tikzset{
  basic/.style  = {draw, rectangle},
  root/.style   = {basic, rounded corners=2pt, thin, align=center,
                   fill=green!30},
  level 2/.style = {basic, rounded corners=6pt, thin,align=center, fill=green!60},
  level 3/.style = {basic, thin, align=left, fill=pink!60}
}

\tikzset{font=\footnotesize}

\definecolor{mygreen}{rgb}{0,0.6,0}
\definecolor{mygray}{rgb}{0.5,0.5,0.5}
\definecolor{mymauve}{rgb}{0.58,0,0.82}
\lstset{ %
	float=ht,
	backgroundcolor=\color{white},   % choose the background color; you must add \usepackage{color} or \usepackage{xcolor}
	basicstyle=\footnotesize\ttfamily,        % the size of the fonts that are used for the code
	breakatwhitespace=false,         % sets if automatic breaks should only happen at whitespace
	breaklines=true,                 % sets automatic line breaking
	captionpos=b,                    % sets the caption-position to bottom
	commentstyle=\color{mygreen},    % comment style
	deletekeywords={...},            % if you want to delete keywords from the given language
	escapeinside={(*@}{@*)},          % if you want to add LaTeX within your code
	extendedchars=true,              % lets you use non-ASCII characters; for 8-bits encodings only, does not work with UTF-8
	frame=tb,	                   % adds a frame around the code
	keepspaces=true,                 % keeps spaces in text, useful for keeping indentation of code (possibly needs columns=flexible)
	keywordstyle=\color{blue},       % keyword style
	%language=C,                     % the language of the code
	otherkeywords={...},           % if you want to add more keywords to the set
	numbers=left,                    % where to put the line-numbers; possible values are (none, left, right)
	numbersep=5pt,                   % how far the line-numbers are from the code
	numberstyle=\tiny\color{mygray}, % the style that is used for the line-numbers
	rulecolor=\color{black},         % if not set, the frame-color may be changed on line-breaks within not-black text (e.g. comments (green here))
	showspaces=false,                % show spaces everywhere adding particular underscores; it overrides 'showstringspaces'
	showstringspaces=false,          % underline spaces within strings only
	showtabs=false,                  % show tabs within strings adding particular underscores
	stepnumber=2,                    % the step between two line-numbers. If it's 1, each line will be numbered
	stringstyle=\color{mymauve},     % string literal style
	tabsize=2,	                   % sets default tabsize to 2 spaces
	%title=\centering\lstname                   % show the filename of files included with \lstinputlisting; also try caption instead of title
}

